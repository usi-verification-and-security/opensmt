\documentclass{article}

\title{The OpenSMT2 Solver}
\author{
Martin Blicha \and 
Antti E. J. Hyv{\"a}rinen \and
Matteo Marescotti \and
Natasha Sharygina \\
{\small Universit{\`a} della Svizzera italiana, Lugano}
}
\date{}
\begin{document}
\maketitle

\section{Overview}
OpenSMT2~\cite{HyvarinenMAS16} is a lazy~\cite{NieuwenhuisOT:JACM06}
DPLL($T$)-based~\cite{DavisLL:ACM1962}
SMT solver written in C{\tt ++}, supporting
the quantifier-free theories of uninterpreted functions, linear real arithmetic, and linear integer arithmetic.
Furthermore, OpenSMT2 supports a wide range of interpolation options, providing users extraordinary customisation opportunities for adaptation to different domains.
Developers can interface OpenSMT2 through the API library exposing most of the internal features. The clear object-oriented structure of the internal design encourages an easy extension to e.g. support more theories and interpolation algorithms.

\section{Utilization}
OpenSMT2 is used in various research work. We mention,
in chronological order, work on interpolation 
algorithms~\cite{BlichaHKS19,AltHAS17,JancikAFHKS16,AsadiBFHESC18}
and parallel SMT 
solving~\cite{HyvarinenMSCS18,MarescottiHS18,HyvarinenMS:SAT15}.
OpenSMT2 is
used as the back-end in model-checking tools
HiFrog~\cite{AltACMFHS17},
eVolCheck~\cite{FSS_TACAS13}, 
FunFrog~\cite{SFS_ATVA12}, and
PeRIPLO~\cite{RolliniAFHS:LPAR2013,AltFHS:VSTTE2015}.
OpenSMT2 is a supported engine in the parallel 
solving framework SMTS~\cite{MarescottiHS16}.

\section{Acknowledgements}
We thank everyone who helped
developing OpenSMT2. In particular,
Leonardo Alt,
Sepideh Asadi,
Martin Blicha,
Roberto Bruttomesso,
Antti E. J. Hyv{\"a}rinen,
Matteo Marescotti,
Edgar Pek,
Simone Fulvio Rollini, 
Parvin Sadigova,
Natasha Sharygina,
Aliaksei Tsitovich.

\bibliography{abstract}
\bibliographystyle{plain}

\end{document}